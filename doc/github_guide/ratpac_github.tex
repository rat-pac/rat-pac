\documentclass{article}
\usepackage[top=1in,left=1in,right=1in,bottom=1in]{geometry}
\usepackage{graphicx}
\usepackage{hyperref}
\usepackage{pdfpages}
\title{Using GitHub with RAT-PAC Code}
\author{Andy Mastbaum\footnote{\href{mailto:mastbaum@hep.upenn.edu}{mastbaum@hep.upenn.edu}}}
\begin{document}
\maketitle
\tableofcontents
\section{Introduction}
Using Git and GitHub makes it easier to share code and allows review to happen
before code is committed to the main repository, ensuring that a high-quality
code base is always available to collaborators. This document introduces Git
and GitHub and provides a walk-through of typical developer workflow. A
``cheat sheet'' suitable for hanging near a computer is attached at the end.

\subsection{What is Git?}
\href{http://git-scm.com}{Git} is a distributed version control system,
originally developed by Linus Torvalds and used for Linux kernel development.
Now, it is used by such prominent projects as Perl, Gnome, Android, X11, and
RAT-PAC.

Distributed version control systems (DVCS) are next-generation version control
systems (VCS) structured with a network of peer repositories, rather than a
single ``master'' repository. In SVN (a regular VCS) for example, users
``check out'' a snapshot of a repository, make changes, and push the new
snapshot to the master repository every time they wish to change something.
In Git (DVCS), users ``clone'' a repository, creating a complete replica of it
on the user's computer. The user can commit to this repository at will, view
the commit history and ``check out'' any past revision offline, and later
synchronize their changes with the parent repository.

This model allows great flexibility in the flow of code changes. It makes it
easier for developers to collaborate on features, and also makes it possible
to create a ``staging area'' for code review.

\subsection{What is GitHub?}
\href{http://www.github.com}{GitHub} is a Git repository hosting service that
offers a suite of best-in-class source code management tools. GitHub
encourages cloning (called {\it forking}) of projects for collaborative
development.
In the past, fork meant divergence: a group of developers broke off from the
project to pursue different direction. Now, thanks to Git's distributed
nature, forking means collaboration: you see a project that needs help,
fork it, make changes, and give those changes back to the original repository.
GitHub makes this process as easy as a few clicks.

The most important feature for RAT-PAC is the {\it Pull Request} system, the
mechanism by which developers ``give those changes back'' from a fork to the
parent repository. Pull Requests provide a chance for code to be reviewed,
discussed, and improved before it is committed to the main repository.

GitHub also offers web-based code browsing, search, and project management
tools including a bug tracking system that integrates with release management.

\section{User's Guide}
This section explains how to get started with RAT development. If you just
need to download RAT on a computer, see Section \ref{sec:install} for a
shortcut.

\subsection{Getting Started}
\label{sec:getting-started}
RAT-PAC is publically-accessible, open-source software, so you don't need a
GitHub account for read-only access. Write access is available to members
of collaborations affiliated with RAT-PAC development. If you are such a
collaborator, you may request access as follows:

\begin{enumerate}
\item Make yourself a free account here:
      \href{https://github.com/signup/free}{https://github.com/signup/free}
\item Send a message to Gabriel and Andy asking to be added to the rat-pac
      organization. Include your full name, GitHub username, insititution,
      and experimental affiliation, and specify that you agree to the policies
      as laid out in the shared development plan document.
\item Once you've been added, you will have write access to the
      repositories at \href{http://github.com/rat-pac}{github.com/rat-pac}.
\end{enumerate}

Now you can browse the code on GitHub, authenticating using your username
and password. But to access the code via the command line, e.g. to download it,
your Git client will need to authenticate too. This is best done with SSH keys.

If you are familiar with SSH keys and already have one, just to
\href{https://github.com/settings/ssh}{https://github.com/settings/ssh},
click ``Add SSH Key'', and paste in your public key. Otherwise, GitHub has a
quick tutorial on how to set this up:
\href{https://help.github.com/articles/generating-ssh-keys}{https://help.github.com/articles/generating-ssh-keys}.

\subsection{Creating a Fork}
\label{create-fork}
Now, you need to create a {\it fork} of the main RAT repository. Your fork is
a duplicate of the main repository which you can work on in isolation,
even doing your own project management with bug tickets and milestones.
When you're ready, you can send a Pull Request, asking to have your changes
merged into the main repository. Typically you will make many commits,
incrementally working toward a new feature, between Pull Requests. This
helps you keep track of every step (and rewind if things go wrong!)

Your fork is a proper repository unto itself. Other people can fork it,
make changes, and send {\it you} Pull Requests, which you can review and
accept, then pass them along as part of a Pull Request to the main RAT
repository. The collaborative possibilities are endless, and in every case,
Git keeps track of all the history of who changed what, when, and why.\\

To fork a repository (be it RAT or any other),
navigate there on GitHub, and click the ``Fork'' button in the upper right
corner. Your fork will now appear in your list of repositories on your GitHub
page (github.com/[your-username]). You should only ever do this once per
repository: work is done in many {\it branches} within a single forked
repository, rather than in many forks.\\

After you click the fork button, your fork only exists on GitHub's servers,
and is identical to the main RAT-PAC repository until the RAT-PAC repository
changes -- this is never propagated automatically!

\subsection{Cloning Your Fork}
\label{sec:manual-install}
The command to clone a repository is\footnote{For complete usage,
try {\tt git help clone}. {\tt git help X} always brings up the man page for
subcommand {\tt X}.}:

\begin{verbatim}
  $ git clone URL [optional destination]
\end{verbatim}
In this case, you'll want\footnote{The URL you need to {\tt git clone} to
clone a repository always appears at the top of that repository's page on
GitHub.}:
\begin{verbatim}
  $ git clone git@github.com:[your-username]/rat.git
\end{verbatim}
This puts RAT in a new directory called {\tt rat}. To put it somewhere else,
tack the destination directory name onto the end of that line. If there is a
particular branch you are interested in, do a
\begin{verbatim}
  $ git fetch
  $ git checkout BRANCH_NAME
\end{verbatim}
to switch to it.\\

To install and run RAT after running {\tt git clone},
you'll need to set up ROOT and GEANT4 as well, source their respective
environment scripts ({\tt \$ROOTSYS/bin/thisroot.sh} and\\
{\tt \$G4INSTALL/bin/geant4.sh}), then run (in the rat directory):
\begin{verbatim}
  $ ./configure
  $ source env.sh
  $ scons
\end{verbatim}
to build RAT.

\paragraph{Don't Forget Your Roots}
You'll want to pull in the changes from the main RAT repository periodically,
to stay up-to-date and avoid conflicts when the time comes to contribute your
changes. To do this, we need to set up a pointer to main repository -- a
remote repository or {\it remote}. The basic syntax is:

\begin{verbatim}
  $ git remote add NAME URL
\end{verbatim}
where {\tt NAME} can be anything you like, and {\tt URL} is a Git URL like the
one we cloned. Here, use:
\begin{verbatim}
  $ git remote add upstream git@github.com:rat-pac/rat-pac.git
\end{verbatim}
This makes a pointer to the main RAT repository and calls it `upstream' (this
is conventional, but has no special meaning to Git).\\

You can also add other peoples' forks as remotes in the same way to pull in
their changes. The command {\tt git remote -v} prints a listing of all the
remotes.\\

The remote that you used in the initial {\tt git clone} command has a name,
too: {\tt origin}.

\subsection{Code Management}
Your fork is yours to manage as you like, but following a few guidelines will
make things much smoother when interacting with the main RAT repository.

\subsubsection{Never Commit Changes to Your `master' Branch} The default ``main''
branch in a Git repository is called ``master,'' and it is strongly suggested
that you keep your master branch in sync with the upstream master branch.
Instead
of modifying master, create {\it topic branches} and do your work there.
Creating branches in Git is simple; to create a new branch and make it active:

\begin{verbatim}
  $ git checkout -b BRANCH-NAME
\end{verbatim}
Try to keep topic branches independent -- avoid developing one branch that
depends on another if possible. It will cause headaches if the base branch is
not accepted into main RAT or requires substantial changes.

\subsubsection{Merge Upstream Changes Often}
\label{sec:merge-upstream}
Development in the main repository may happen
fast, and if you work in isolation for a long time, you will find that
the changes you've made are no longer compatible with the main version of RAT
and cannot be merged in. To prevent this, stay up to date and resolve small
conflicts as they occur. If you set up the
`upstream' remote as outlined above, run:

\begin{verbatim}
  $ git fetch upstream         # grab the changes from the remote
  $ git merge upstream/master  # merge changes into the current branch
\end{verbatim}

\paragraph{Handling Conflicts} 
It is possible that changes have been made to the main repository that
conflict with your local changes. Git makes an effort to resolve conflicts
automatically, but in some cases (like edits to the same line), it can't
tell what is correct. In this case, {\tt git merge} will keep both versions,
so that the files look like this:

\begin{verbatim}
<<<<<<<
Version in the current branch (yours)
=======
Version in the branch being merged in (theirs)
>>>>>>>
\end{verbatim}
It will tell you which files have conflicts. Go and fix them by hand, choosing
the correct version or editing as necessary, then mark the conflict as
resolved by running {\tt git add FILENAME}. Once all conflicts are resolved,
you need to complete the merge by committing:

\begin{verbatim}
  $ git commit -am "merge branch upstream/master"
\end{verbatim}
{\it Remember: just because Git conflicts are resolved does not mean that the
code will actually compile and run! Be sure to check before
committing your merge.}\\

There are other ways of handling merge conflicts -- see {\tt git help merge}
for information.

\subsection{Viewing Status and Changes}
\label{sec:status}
Git has a few commands to keep track of changes:

\begin{description}
\item[{\tt git status}] Shows which files have been modified, added, and removed
\item[{\tt git diff}] Shows a summary of all changes since the last commit
\item[{\tt git diff X}] Shows outstanding changes relative to {\it ref} X
\item[{\tt git diff X Y}] Shows differences between {\it ref} X and Y
\end{description}
A {\it ref} in Git is a pointer to a commit. This can be a revision ID (the long
hexadecimal identifier), a tag name (like {\tt 1.0} for an imaginary RAT 1.0),
a branch
name (like {\tt master}), or one of a handful of special identifiers (like
{\tt HEAD$\sim$5} for five commits ago).

\subsection{Committing Changes}
A {\it commit} packages all of the outstanding changes into a {\it changeset}
which becomes part of the history of the repository. Each commit is
accompanied with a {\it message} -- a description of what the changes are and
why they were made.

Git only records changes for files when it's told to, so if you have created
new files you need to do:

\begin{verbatim}
  $ git add FILENAME
\end{verbatim}
to inform Git that this file should be tracked.\\

Now, to commit the changes (including newly-added files and modifications to
existing ones):

\begin{verbatim}
  $ git commit -a
\end{verbatim}
Remember not to commit to `master'! The {\tt -a} here means to commit {\bf all}
the outstanding changes shown by {\tt git status}. To instead commit
individual files, use {\tt git commit file1 [file2 ...]}.\\

This command will pop open your system's default editor so you can type in a
commit message (see Section \ref{sec:best-practices} for best practices).
To change the editor, set the {\tt EDITOR} environment variable, e.g.
{\tt \$ export EDITOR=vim}. For a short message, you can also do
{\tt \$ git commit -am "message here"}.\\

Now, the changes are stored in the repository on your computer. Periodically,
you should {\it push} your changes to the GitHub server, which will serve as a
backup and also let people see what you're working on. Eventually, all changes
must be pushed to GitHub in order to be merged into the main repository, but
you don't need to push after every commit. To push changes, the command is:

\begin{verbatim}
  $ git push [REMOTE NAME] [BRANCH NAME]
\end{verbatim}
So if you are working on a topic branch called `awesome-feature', use:

\begin{verbatim}
  $ git push origin awesome-feature
\end{verbatim}
This creates and populates the branch `awesome-feature' on GitHub's servers
if it doesn't already exist. If you run {\tt git push} with no arguments, Git
will push all the branches that match existing remote branches\footnote{You
can configure the default behavior with {\tt git config}.}. It's probably
best to just specify explicitly the branch you want to push.

Remember that {\tt origin} is the name of the remote used in {\tt git clone},
i.e. your fork.

\subsection{Documenting Changes}
RAT features should be documented in the RAT User's Guide.
When you submit a pull request to the RAT repository that adds a new feature
or changes the way RAT works, you should always be sure to update or add to
the documentation in the same pull request. Pull requests that neglect the
documentation are likely to be rejected!

This documentation, written in Sphinx-style ReStructured Text (ReST), lives in
the {\tt doc} subdirectory of the rat-pac repository. For help with writing in
ReST, see \href{http://sphix-doc.org}{http://sphinx-doc.org}.

The User's Guide is compiled into web-accessible HTML after each commit to
the main RAT-PAC repository; this can be found at
\href{http://rat.rtfd.org}{http://rat.rtfd.org}.

\subsection{Best Practices}
\label{sec:best-practices}
There are also a few tips to follow to maximize your developer karma:

\subsubsection{Write Good Commit Messages} The commit message should record
what was changed and why. The proper format for a Git commit message is:

\begin{verbatim}
Describe changes briefly (50 chars or less)

After a blank line, provide a detailed, multi-line description,
wrapped to 72 characters.

 - Bullet points are nice
 - ... if you have things to list
\end{verbatim}
GitHub will format these messages nicely, and Git fanatics are {\it rabid}
about commit message formatting. Also note the use of present-tense verbs:
{\it edit} the file, {\it fix} the bug.

\subsubsection{Keep the History Tidy} On one hand, the
history should be a complete record, but on the other hand, extraneous changes
(e.g. changed thing, reverted change, reverted revert, ...) clutter the
history and make it very difficult to follow.

As a guideline, a typical pull request should have one or a few commits. If
you prefer finer granularity (lots of tiny commits) when developing, use Git's
``squash'' merge to combine your changes into a single changeset with a
meaningful message. Say you've developed a feature on branch `mybranch':

\begin{verbatim}
  $ git fetch upstream
  $ git checkout -b mybranch-squash upstream/master  # make new branch from the latest master
  $ git merge --squash mybranch  # merge in your changes all at once
  $ git commit  # commit the merge. the editor will give you the list of individual messages
\end{verbatim}

\subsubsection{Don't Rebase} Git offers the powerful but deadly {\tt git rebase}
command which can modify the Git history. This will in general
render the rebased branch incompatible with other branches, and should be
avoided unless you are certain that the affected commits aren't in use by
anyone else.

\subsection{Contributing Your Changes}
At last, the feature is done, the bug is fixed, the document is documented;
you're ready to push your changes off into the main repository.
First, consider:

\begin{itemize}
\item Is the change appropriately documented?
\item Does the code comply with the style conventions of RAT?
\item Does the code introduce any new external dependencies?
\item Have any relevant {\tt rattests}s been updated?
\end{itemize}

Push all of your changes to a branch on your GitHub fork. Then to send it off
for review, click the ``Pull Request'' button at the upper
right corner of your fork's repository page on GitHub. On the Pull Request
page, you can choose the correct topic branch and review the changes.
Enter a title and description of your changes and click
``Send pull request.''

This will notify everyone following the the main repository, and someone will
take a look over the changes. If you're contributing an experiment-specific
feature, ask a qualified person to do the review and merge the changes.

If anything's amiss, such as missing documentation, the reviewer can make
comments, and you can add commits to address any issues. The pull request
page becomes a conversation about the changes, where people can make
comments on individual commits, lines, or blocks of code to address any issues.

\section{Tips}
\subsection{Just Downloading RAT}
\label{sec:install}
Once SSH keys are set up, getting RAT is a one-liner:
\begin{verbatim}
  $ git clone git@github.com:rat-pac/rat-pac.git
\end{verbatim}
This puts RAT in a new directory called {\tt rat}. To put it somewhere else,
tack the destination directory name onto the end of that line. You can also
download RAT over HTTPS (substitute the URL
https://github.com/rat-pac/rat-pac), without
involving SSH keys at all.

To update with changes from the main repository, run {\tt git pull}.\\

If you change your mind and want to do development in this directory, you can
follow the instructions in Section \ref{sec:manual-install}.

\subsection{Checking Out an Old Revision}
When you clone a repository, you have the entire history on your computer,
so you can check out any revision, even while on a plane. Run
{\tt git checkout} to get to another revision:

\begin{verbatim}
  $ git checkout X
\end{verbatim}
where {\tt X} is a {\it ref}, as explained in Section \ref{sec:status}.
After running this, your directory is magically transformed into the revision
you asked for. If this is a RAT repository and you want to run this version,
you'll need to re-build:

\begin{verbatim}
  $ ./configure
  $ source ./env.sh
  $ scons -c
  $ scons
\end{verbatim}
If you want to keep several different versions of RAT compiled at once, the
easiest way to keep them in separate directories.

\subsection{Finding a Commit}
Because the history of a Git repository doesn't have to be linear, revisions
can't be named with a sequential number. You can find a revision ID by:
\begin{itemize}
\item Searching the history on GitHub:
\href{https://github.com/rat-pac/rat-pac/search}{https://github.com/rat-pac/rat-pac/search}.
\item Looking at the history on github.com: \href{https://github.com/rat-pac/rat-pac/commits/master}{https://github.com/rat-pac/rat-pac/commits/master}
\item Looking at the history by running {\tt git log}: This opens in {\tt less}, so you can search.
\end{itemize}

\subsection{Fixing Common Commit Mistakes}
If you made a mistake in a commit but haven't pushed yet, there is
still time to fix it. This will change the ID of the commit, so don't do this
with commits that have been pushed -- fix it in a new commit.\\

You can easily fix a bad author name or email (these should match your
GitHub user information):
\begin{verbatim}
  $ git commit --amend --author="Your Name <email@website.com>"
\end{verbatim}
Or a mistake in a message:
\begin{verbatim}
  $ git commit --amend -m "Fixed message"
\end{verbatim}
Or add a forgotten file:
\begin{verbatim}
  $ git add forgotten_file.cpp
  $ git commit --amend
\end{verbatim}

\subsection{Accidental Changes to a Branch}
Inevitably, in your zeal to code, you will start editing files and realize
you've forgotten to make a topic branch, so you are editing `master' or
another branch you didn't intend to modify. Do not despair; Git has your back.

\begin{verbatim}
  $ git stash  # "stash" your outstanding changes
  $ git checkout -b BRANCH-NAME  # make that topic branch
  $ git stash pop  # apply the stashed changes
\end{verbatim}
You're back in business.

\subsection{Accidental Commits to a Branch}
If you accidentally commit changes to the wrong branch (like `master'), you can
still fix it.

\begin{verbatim}
  $ git checkout -b NEW-BRANCH-NAME  # make a topic branch based from the current branch
  $ git checkout MESSED-UP-BRANCH-NAME  # go back to the messed up branch
  $ git reset --hard HEAD^  # reset state to the previous commit
\end{verbatim}
If you wanted to apply the changes to a different, existing branch, see
{\tt git help cherry-pick} for a tool to do this.

\subsection{When a Pull Request Has Merge Conflicts}
When you submit a pull request, you may be asked to merge in the main RAT
`master' branch. Follow the instructions in Section \ref{sec:merge-upstream}:
Git will identify the conflicts for you. Fix them, commit the merge, and
push the branch to GitHub to update the pull request.

\section{Git and GitHub Resources}
A few of the great resources for getting acquainted with Git:
\begin{description}
\item[GitHub Bootcamp] \href{http://help.github.com}{help.github.com}
\item[Git Cheat Sheets] \href{http://help.github.com/git-cheat-sheets}{help.github.com/git-cheat-sheets}
\item[Complete Git command reference] \href{http://gitref.org}{gitref.org}
\item[{\tt git help}] Git is very well-documented -- try, e.g. {\tt git help branch}.
\end{description}

\end{document}

